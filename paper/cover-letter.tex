\documentclass{letter}
\usepackage[a4paper, total={5.2in, 9.4in}]{geometry}
\usepackage{setspace}
\signature{Fernando Cagua \\ (on behalf of all authors)}
\address{Centre for Integrative Ecology \\ School of Biological Sciences \\ University of Canterbury \\ Private Bag 4800 \\ 8140 Christchurch \\ New Zealand}
\begin{document}
\begin{letter}{James Ross \\ Assistant Editor \\ Journal of Ecology}
\opening{Dear James:}

\onehalfspacing

We are thankful to the senior and associate editors, Prof.\ David Gibson and Prof.\ Nicole Rafferty, and two anonymous reviewers for the insightful comments to improve the manuscript. 
We are also grateful for the opportunity to resubmit. 

In the attached document, you can find a detailed response to each of their comments. 
For your benefit, we also summarise here the main changes we have made to our manuscript. 

Briefly, our paper explores the possible applications of control theory in an ecological context. 
We honestly see potential in our paper to contribute to both applied and theoretical aspects in ecology. 
However, our initial submission was caught between these two directions and---as the editors and reviewers rightly pointed out---we failed to satisfactorily address either option as a result.
We have re-scrutinised our manuscript and concluded that the theoretical insights are stronger. 
As a consequence, we have chosen to focus on the conceptual aspect of our approach in this revision at the expense of numerous, but weaker, connections with conservation and applied ecology.

Adopting a primarily conceptual focus has allowed us to be clearer about the justification, methodology and potential insights of our approach. 
However, improving clarity while successfully tying the conceptual motivation with our findings required more than some changes to some paragraphs or sections of the manuscript. 
Therefore, we have rewritten almost entirely the manuscript and the supplementary materials. 
We maintain the methodology and findings, but there has been a major change in the way we justify the study, introduce novel concepts to the ecological domain, emphasize results, and present our conclusions. 

We look forward to further feedback and hope that the editors and reviewers find these changes satisfactory.

\closing{Regards,}
\end{letter}
\end{document}
