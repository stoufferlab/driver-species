\documentclass[a4paper]{artikel1}
\usepackage{lmodern}
\usepackage{amssymb,amsmath}
\usepackage{ifxetex,ifluatex}
\usepackage{fixltx2e} % provides \textsubscript
\ifnum 0\ifxetex 1\fi\ifluatex 1\fi=0 % if pdftex
  \usepackage[T1]{fontenc}
  \usepackage[utf8]{inputenc}
\else % if luatex or xelatex
  \ifxetex
    \usepackage{mathspec}
  \else
    \usepackage{fontspec}
  \fi
  \defaultfontfeatures{Ligatures=TeX,Scale=MatchLowercase}
\fi
% use upquote if available, for straight quotes in verbatim environments
\IfFileExists{upquote.sty}{\usepackage{upquote}}{}
% use microtype if available
\IfFileExists{microtype.sty}{%
\usepackage{microtype}
\UseMicrotypeSet[protrusion]{basicmath} % disable protrusion for tt fonts
}{}
\usepackage[margin=1in]{geometry}
\usepackage{hyperref}
\hypersetup{unicode=true,
            pdftitle={Keystoneness, centrality, and the structural controllability of ecological networks},
            pdfauthor={Fernando Cagua, Kate L. Wootton, Daniel B. Stouffer},
            pdfborder={0 0 0},
            breaklinks=true}
\urlstyle{same}  % don't use monospace font for urls
\usepackage{longtable,booktabs}
\usepackage{graphicx,grffile}
\makeatletter
\def\maxwidth{\ifdim\Gin@nat@width>\linewidth\linewidth\else\Gin@nat@width\fi}
\def\maxheight{\ifdim\Gin@nat@height>\textheight\textheight\else\Gin@nat@height\fi}
\makeatother
% Scale images if necessary, so that they will not overflow the page
% margins by default, and it is still possible to overwrite the defaults
% using explicit options in \includegraphics[width, height, ...]{}
\setkeys{Gin}{width=\maxwidth,height=\maxheight,keepaspectratio}
\IfFileExists{parskip.sty}{%
\usepackage{parskip}
}{% else
\setlength{\parindent}{0pt}
\setlength{\parskip}{6pt plus 2pt minus 1pt}
}
\setlength{\emergencystretch}{3em}  % prevent overfull lines
\providecommand{\tightlist}{%
  \setlength{\itemsep}{0pt}\setlength{\parskip}{0pt}}
\setcounter{secnumdepth}{5}
% Redefines (sub)paragraphs to behave more like sections
\ifx\paragraph\undefined\else
\let\oldparagraph\paragraph
\renewcommand{\paragraph}[1]{\oldparagraph{#1}\mbox{}}
\fi
\ifx\subparagraph\undefined\else
\let\oldsubparagraph\subparagraph
\renewcommand{\subparagraph}[1]{\oldsubparagraph{#1}\mbox{}}
\fi

%%% Use protect on footnotes to avoid problems with footnotes in titles
\let\rmarkdownfootnote\footnote%
\def\footnote{\protect\rmarkdownfootnote}

%%% Change title format to be more compact
\usepackage{titling}

% Create subtitle command for use in maketitle
\newcommand{\subtitle}[1]{
  \posttitle{
    \begin{center}\large#1\end{center}
    }
}

\setlength{\droptitle}{-2em}
  \title{Keystoneness, centrality, and the structural controllability of
ecological networks}
  \pretitle{\vspace{\droptitle}\centering\huge}
  \posttitle{\par}
\subtitle{Supporting Information}
  \author{Fernando Cagua, Kate L. Wootton, Daniel B. Stouffer}
  \preauthor{\centering\large\emph}
  \postauthor{\par}
  \date{}
  \predate{}\postdate{}

\usepackage{booktabs}
\usepackage{setspace}
\usepackage{xr}
\externaldocument[M-]{manuscript}
\usepackage{multirow}

\usepackage{amsthm}
\newtheorem{theorem}{Theorem}[section]
\newtheorem{lemma}{Lemma}[section]
\theoremstyle{definition}
\newtheorem{definition}{Definition}[section]
\newtheorem{corollary}{Corollary}[section]
\newtheorem{proposition}{Proposition}[section]
\theoremstyle{definition}
\newtheorem{example}{Example}[section]
\theoremstyle{definition}
\newtheorem{exercise}{Exercise}[section]
\theoremstyle{remark}
\newtheorem*{remark}{Remark}
\newtheorem*{solution}{Solution}
\begin{document}
\maketitle

\renewcommand\thefigure{S\arabic{figure}}    

\setcounter{figure}{0}

\renewcommand\thetable{S\arabic{table}}    

\setcounter{table}{0}

\renewcommand{\thesection}{S\arabic{section}}

\setcounter{section}{0}

\doublespacing

\section{Characteristics of the empirical
networks}\label{empirical-networks}

The networks studied had species richness ranging between 19 and 87
(when considering only the largest component in each network). As shown
by the network asymmetry \emph{AS} (N. Blüthgen, Menzel, Hovestadt,
Fiala, \& Blüthgen, 2007), the networks had a low ratio of plants to
pollinators overall. Furthermore, the networks had relatively low levels
of nestedness (when measured using the quantitative version of the NODF
index; Almeida-Neto \& Ulrich, 2011). Details for each network can be
found in the Table \ref{tab:table-network-properties}.

\begin{table}

\caption{\label{tab:table-network-properties}Properties of the analysed plant-pollinator communities. Here show the number of species ($n_s$), the number of plants ($n_p$), the number of pollinators ($n_a$), the network connectance ($C$), the network asymmetry ($AS$), and the network nestedness (NODF index). All properties correspond to the network's largest component. British networks were assembled by Lopezaraiza-Mikel et al. (2007), Spanish were networks assembled by Bartomeus et al. (2008).}
\centering
\fontsize{8}{10}\selectfont
\begin{tabular}[t]{rlrrrrrrl}
\toprule
site & invader & $n_s$ & $n_p$ & $n_a$ & $c$ & $AS$ & NODF & location\\
\midrule
1 & — & 35 & 9 & 26 & 0.17 & -0.49 & 8.68 & Cap de Creus, Spain\\
1 & \textit{Carpobrotus affine} & 57 & 10 & 47 & 0.17 & -0.65 & 13.27 & Cap de Creus, Spain\\
\addlinespace
2 & — & 40 & 10 & 30 & 0.16 & -0.50 & 11.66 & Cap de Creus, Spain\\
2 & \textit{Carpobrotus affine} & 38 & 11 & 27 & 0.21 & -0.42 & 15.04 & Cap de Creus, Spain\\
\addlinespace
3 & — & 31 & 7 & 24 & 0.19 & -0.55 & 12.91 & Cap de Creus, Spain\\
3 & \textit{Opuntia stricta} & 33 & 8 & 25 & 0.18 & -0.52 & 9.96 & Cap de Creus, Spain\\
\addlinespace
4 & — & 35 & 10 & 25 & 0.17 & -0.43 & 12.43 & Cap de Creus, Spain\\
4 & \textit{Carpobrotus affine} & 57 & 14 & 43 & 0.14 & -0.51 & 13.70 & Cap de Creus, Spain\\
\addlinespace
5 & — & 35 & 8 & 27 & 0.19 & -0.54 & 11.91 & Cap de Creus, Spain\\
5 & \textit{Opuntia stricta} & 32 & 8 & 24 & 0.19 & -0.50 & 10.96 & Cap de Creus, Spain\\
\addlinespace
6 & — & 30 & 9 & 21 & 0.17 & -0.40 & 6.91 & Cap de Creus, Spain\\
6 & \textit{Opuntia stricta} & 37 & 9 & 28 & 0.17 & -0.51 & 12.45 & Cap de Creus, Spain\\
\addlinespace
7 & — & 37 & 6 & 31 & 0.19 & -0.68 & 18.33 & Bristol, United Kingdom\\
7 & \textit{Impatients grandulifera} & 57 & 8 & 49 & 0.20 & -0.72 & 14.36 & Bristol, United Kingdom\\
\addlinespace
8 & — & 48 & 5 & 43 & 0.21 & -0.79 & 5.87 & Bristol, United Kingdom\\
8 & \textit{Impatients grandulifera} & 87 & 15 & 72 & 0.11 & -0.66 & 8.12 & Bristol, United Kingdom\\
\addlinespace
9 & — & 55 & 12 & 43 & 0.13 & -0.56 & 13.01 & Bristol, United Kingdom\\
9 & \textit{Impatients grandulifera} & 86 & 11 & 75 & 0.13 & -0.74 & 13.40 & Bristol, United Kingdom\\
\addlinespace
10 & — & 19 & 3 & 16 & 0.38 & -0.68 & 9.99 & Bristol, United Kingdom\\
10 & \textit{Impatients grandulifera} & 54 & 6 & 48 & 0.21 & -0.78 & 7.56 & Bristol, United Kingdom\\
\bottomrule
\end{tabular}
\end{table}

\clearpage

\section{Structural controllability}\label{structural-controllability}

Structural controllability avoids the limitation of not knowing the
exact values of the matrices \emph{A} and \emph{B}. Structural
controllability boils down to two conditions: a system is controllable
if there are no inaccessible nodes or dilations. A node is inaccessible
if there are no directed paths between it and the input nodes.
``Dilations are subgraphs in which a small subset of nodes attempts to
rule a larger subset of nodes. In other words, there are more
`subordinates' than `superiors'\,'' (Liu \& Barabási, 2016).

The goal of structural controllability is to use the information
contained in \textbf{A} to generate a supportable estimate of
\textbf{B}. This focus allows us to gain insight of the inherent
controllability of a network, and the roles of the species that compose
it, without being overly dependent on the particular choices of how the
system dynamics are modelled or characterised. The trade-off of this
approach is that, because of the assumption of linearity, structural
controllability alone does not allow us to fully design the time-varying
control signal \emph{u(t)} that can drive the system from one particular
equilibrium to another. Nevertheless, the lessons gained when assuming
linearity---at both the network and the species level---are a
prerequisite for eventually understanding nonlinear control (Liu \&
Barabási, 2016; Liu, Slotine, \& Barabási, 2011).

\newpage

\section{Maximum matching}\label{maximum-matching}

\subsection{Finding a single maximum
matching}\label{one-maximum-matching}

Our approach to finding the minimum number of driver nodes relies on
finding maximum matchings. We start with a directed network in which the
direction of the link represents the direction of control (Figure
\ref{fig:fig-maximum-matching-procedure} left panel). We then construct
an alternative representation of the directed network in which each node
of the directed network is represented by two nodes that indicate their
outgoing and incoming links respectively (Figure
\ref{fig:fig-maximum-matching-procedure} centre panel). Finding a
maximum matching in this alternative representation is equivalent to
finding the largest possible set of edges in which one node on the
left-hand side is connected to at most one node on the right-hand side.
To find the maximum matching we use the push-relabel algorithm
implemented in \texttt{max\_bipartite\_matching} in the R package igraph
1.0.1 (Csardi \& Nepusz, 2006). Once we have the matching (shown in the
Figure \ref{fig:fig-maximum-matching-procedure} center panel) it is then
easy to identify the roles of each node in this representation: nodes on
the top-level that are connected to a matched link (dark purple) are
superior while those connected to a matched link on the bottom-level are
matched. This information can then be mapped back to the original
representation to identify the control paths and the driver nodes in the
network (Figure \ref{fig:fig-maximum-matching-procedure} right panel).

\begin{figure}[p]

{\centering \includegraphics{/Users/efc29/github/driver-species/paper/supp-info_files/figure-latex/fig-maximum-matching-procedure-1} 

}

\caption{Finding a maximum matching in a complex network. (left) Directed network that indicate the direction of control between species. (center) Alternative bipartite representations of the directed networks. (right) The matchings in the bipartite representation mapped back to the original network.}\label{fig:fig-maximum-matching-procedure}
\end{figure}

To further illustrate our methodology here, we also show the approach
for the smallest of our empirical networks, the uninvaded network at
site 10 (Table \ref{tab:table-network-properties}; Figure
\ref{fig:fig-small-network}). This network is composed of 19 species of
which three are non-invasive plants and the other 16 are pollinators.
The one-to-one relationship between matched and superior nodes implies
that in order to achieve full network controllability, most pollinators
would be unmatched, and hence are classified as driver nodes that
require external intervention. At the same time, both plants in the
community, \emph{Heracleum sphondylum} and \emph{Rubus fructicosus}, and
one of the pollinators, \emph{Orthotylus/Lygocorus}, tend to be
classified as superior nodes.

\begin{figure}
\centering
\includegraphics{/Users/efc29/github/driver-species/paper/supp-info_files/figure-latex/fig-small-network-1.pdf}
\caption{\label{fig:fig-small-network}Illustration of the procedure with an
empirical community. The visitation network (top), the number of visits
between species pairs are shown on each link. The directed network in
which the direction of control is determined based on the mutual
dependences (middle). One of the possible maximum matchings of this
network (bottom).}
\end{figure}

\subsection{Finding all possible maximum
matchings}\label{all-maximum-matching}

The algorithm implemented in \texttt{max\_bipartite\_matching}, however,
is only able to find \textbf{one} of possibly many maximum matchings in
a network. Though one maximum matching is enough to calculate
\emph{n\textsubscript{D}} and hence to provide an indication of the
manageability of a community, it is not sufficient to estimate the role
of individual species. To do that, we need to calculate all possible
maximum matchings. To do this, we again start from the alternative
bipartite representation in Figure
\ref{fig:fig-maximum-matching-procedure} (center panel) and assign an
identity to each of the links in the network. We will call this
bipartite representation \emph{P}. We then construct the line graph of
the alternative bipartite representation \emph{L(P)} (Figure
\ref{fig:fig-all-maximum-matching}). Each node in \emph{L(P)} represents
a link in \emph{P} and these are connected to each other if and only if
they share a common node in \emph{P}. We then calculate \emph{H}, the
complement graph of \emph{L(P)} and identify all of its maximal cliques
(Figure \ref{fig:fig-all-maximum-matching}). Here some extra definitions
are necessary. First, \emph{H} is a graph with the same nodes as
\emph{L(P)} but that has a link between two nodes if and only if there
is not a link in \emph{L(P)}. Second, a clique is a subset of nodes such
that all pairs of them are linked. Lastly, a maximal clique is a clique
such that there are no cliques composed of more nodes (Gutin, 2013). In
this example, there are two maximal cliques. The final step is then to
map these cliques onto the original network to obtain all possible
maximal cardinality matchings as shown in Figure
\ref{M-fig:fig-control-configuration} in the main text.

\begin{figure}
\centering
\includegraphics{/Users/efc29/github/driver-species/paper/supp-info_files/figure-latex/fig-all-maximum-matching-1.pdf}
\caption{\label{fig:fig-all-maximum-matching}Finding all possible maximum
matchings. From left to right: alternative bipartite representation of
the directed network. Line graph of the network. The complement of the
network. The two maximal cliques are shown in dark purple.}
\end{figure}

\clearpage

\subsection{Control adjacency and the input graph}\label{input-graph}

The algorithm proposed by Zhang, Lv, \& Pu (2016) enables us to find all
minimum driver node sets without calculating all possible maximum
matchings. The algorithm is based on the construction of the input graph
of the directed network. Components in the input graph reveal the
correlations between nodes from a structural control perspective (Figure
\ref{fig:fig-input-graph}). The input graph is constructed based on the
\emph{control adjacency} of nodes in a maximum matching. Two nodes
\(x_i\) and \(x_j\) are said to be control adjacent if there is a node
\(x_k\) connecting \(x_i\) and \(x_j\) with an unmatched link
\(x_k-x_i\) and a matched link \(x_k-x_j\). For example \(a_1\) is
control adjacent to \(a_2\) in the top panel of Figure
\ref{fig:fig-input-graph}. If a node \(x_i\) is part of a minimum driver
node set then every node in the same component of \(x_i\) in the input
graph is a possible driver node.

\begin{figure}
\centering
\includegraphics{/Users/efc29/github/driver-species/paper/supp-info_files/figure-latex/fig-input-graph-1.pdf}
\caption{\label{fig:fig-input-graph}Input graph. Two nodes \(x_i\) and
\(x_j\) are said to be control adjacent if there is a node \(x_k\)
connecting \(x_i\) and \(x_j\) with an unmatched link \(x_k-x_i\) and a
matched link \(x_k-x_j\). For example, in the top panel, \(a_1\) is
control adjacent to \(a_2\) (via \(p_2\)), \(p_1\) is control adjacent
to \(p_3\) (via \(a_1\)) and \(p_2\) has no control adjacent nodes. As
\(a_1\) and \(p_2\) belong to the minimum driver node set under the
control configuration found with the maximum matching shown, then every
node in the same component as these two nodes is also a possible driver
node. By combining the nodes from the components with minimum driver
node sets is possible to enumerate the different control configurations.
For example the network in the top panel has two possible minimum driver
node sets \([p_2, a_1]\) and \([p_2, a_2]\) (Figure
\ref{M-fig:fig-control-configuration}). The network in the middle panel
has three minimum driver node sets \([x_1, x_2]\), \([x_1, x_3]\), and
\([x_1, x_4]\). Finally the network in the bottom pannel has only one
\([x_1]\).}
\end{figure}

In order to calculate a node's control capacity \(\phi\), we harness the
fact that nodes can be classified into three groups according to their
role in the input graph. First, if \(x_i\) belongs to the minimum driver
node set and has no adjacent nodes, then it cannot be replaced and will
be part of every possible control configuration. Second, if \(x_i\) does
not belong to the minimum driver node set and is not adjacent to any
possible input node, then it does not become part of the minimum driver
node set under any control configuration. Finally, if a node \(x_i\)
belongs to the minimum driver node set and is adjacent to another node
\(x_j\) in the input graph, then \(x_j\) must belong to another minimum
driver node to which \(x_i\) does not belong, in other words, \(x_i\)
and \(x_j\) are substitutable.

The first and the second group have a control capacity of \(\phi = 1\)
and \(\phi = 0\) respectively. Meanwhile, the third group has
intermediate values of control capacity which can be calculated by
computing the number of possible substitutions of a node by its control
adjacent nodes. As a rule of thumb, the larger the number of possible
substitutions an input node has, the smaller its control capacity.

For larger networks, calculating the control capacities using the input
graph is several orders of magnitude faster than computing all possible
maximum matchings. While the computation time of the input graph grows
linearly with network size, the computation of all maximum matchings
grows polynomially. However, the control capacities obtained using the
input graph are not identical to those obtained by computing all
possible maximum matchings. The reason for this discrepancy is that two
maximum matchings might result in the same minimum driver node set.
Nevertheless, the results obtained using both approaches are extremely
similar. We compared the control capacity obtained by calculating all
maximum matchings and that obtained using the input graph in 13 of our
networks (those for which we were able to calculate all maximum
matchings within a reasonable time). The Spearman correlation was very
high in all cases, ranging between 0.89 and 1 (median 0.98).

\clearpage

\subsection{Networks with reciprocal links}\label{reciprocal-links}

Our procedures to find minimum driver node sets are only adequate when
there are no reciprocal (bidirectional) links in the network. The
proportion of reciprocal links in relatively low. In fact, the
proportion of species pairs that are reciprocally connected in our
empirical networks is just 2.97\%. However, reciprocal links are not a
rare occurrence as 16 out of 20 empirical networks had at least one. To
calculate our controllability metrics at both the network (the relative
size of the minimum driver node set, \(n_D\)) and species level (control
capacity, \(\phi\), and wether it is a superior node, \(\sigma\)) we
first need to generate all possible versions of the of the network that
include only non-reciprocal links (Figure \ref{fig:fig-bidirectional}).

All of these non-reciprocal networks are as likely to be one that best
represents the superior/matched relationships in the network. As such,
we average the three metrics (\(n_D\), \(\phi\), and \(\sigma\)) across
all the networks without reciprocal links without weighting by the
number of possible control configurations. It is important to note that
a species that is found to be a superior node under a particular control
configuration, will be so in all possible control configurations for the
particular non-reciprocal network. We, therefore denote the average of
\(\sigma\) across all non-reciprocal versions as the probability of the
species being a superior node in its community.

\begin{figure}
\centering
\includegraphics{/Users/efc29/github/driver-species/paper/supp-info_files/figure-latex/fig-bidirectional-1.pdf}
\caption{\label{fig:fig-bidirectional}In order to calculate the
controllability metrics for a network with reciprocal links (left), we
first decompose the network into several versions that contain no
reciprocal links (right). As each of these networks represents a
different equiprobable scenario, the controllability metrics for the
network with reciprocal links are the unweighted mean of these metrics
across the non-reciprocal networks.}
\end{figure}

\clearpage

\section{Structural stability}\label{structural-stability}

We follow Rohr, Saavedra, \& Bascompte (2014) and Saavedra, Rohr,
Olesen, \& Bascompte (2016) to calculate the structural stability of the
empirical networks. We first calculate the stability condition
\(\hat{\gamma}\) using all species in the network. To calculate the
contribution to stable coexistence of a given species, we removed the
focal species from the network and then calculated the corresponding
structural stability \(\Omega\) for a level of mutualism equal to the
stability condition found previously. For simplicity, in the main text
we show results for when there is a mutualistic trade-off of
\(\delta = 0\) and a mean interspecific competition of \(\rho = 0.01\)
(Saavedra et al., 2016). The choice of the parameters had only a minor
impact on the relative contribution of species to stable coexistence
(Figure \ref{fig:fig-rho-sensitivity}a) and overall there was no change
on our result that indicates that critical species (those with
\(\phi = 1\)) have a larger than average contribution to stable
coexistence (Figure \ref{fig:fig-rho-sensitivity}b).

\begin{figure}[p]

{\centering \includegraphics{/Users/efc29/github/driver-species/paper/supp-info_files/figure-latex/fig-rho-sensitivity-1} 

}

\caption{A sensitivity analysis of the impact of interspecific competition parameter $\rho$ on structural stability reveals that the choice of parameters has a negligible impact on the results. (a) Spearman correlation coefficient of the contribution to stable coexistence of species. Lighter lines indicate the correlation coeficient for each of the twenty empirical networks. The dark line indicates the median value. (b) Difference between the conribution to stable coexistence of critical and redundant species for different values of the iterspecific competition parameters $\rho$.}\label{fig:fig-rho-sensitivity}
\end{figure}

\clearpage 

\section{Visitation as a proxy for species
dependence}\label{visitation-as-proxy}

Visitation frequency has been shown to be an appropriate surrogate for
inter-specific effects in pollination networks (Bascompte, Jordano, \&
Olesen, 2006; Vázquez, Morris, \& Jordano, 2005). Nevertheless,
visitation is not equivalent to pollen deposition and might be
insufficient to reflect the dependences of plants on animals and vice
versa (Alarcón, 2010; King, Ballantyne, \& Willmer, 2013). We,
therefore, investigated the effect of calculating the dependences using
visitation or pollination effectiveness and importance---two metrics
more proximate to plant reproductive success. We did this by comparing
\emph{(i)} the size of the minimum driver node set \(n_D\) of the
community, \emph{(ii)} the species' control capacity and, \emph{(iii)}
the species' probability of being a superior node. To do this, we used
data collected by G. Ballantyne, Baldock, \& Willmer (2015) from a low
diversity pollination community at a dry lowland heathland in Dorset, UK
(50° 43.7'N 2° 07.2'W). First, deposition networks were quantified using
the mean single visit deposition---the number of conspecific pollen
grains effectively deposited on a virgin stigma during a single visit by
a particular animal (G. Ballantyne et al., 2015; King et al., 2013;
Ne'eman, Jürgens, Newstrom-Lloyd, Potts, \& Dafni, 2010). Second,
visitation networks were constructed by counting the visits to flowers
during Single Visit Depositions. Finally, pollinator importance networks
were constructed as the product of pollinator efficiency and visit
frequency.

At a network scale the size of the minimum driver node set \(n_D\) was
consistent among the three weighting schemes (0.67 for deposition, 0.67
for the visitation, and 0.62 for the pollinator-importance network,
respectively). The choice of weighting can also have an impact on at the
species level. Therefore we calculated \(\phi\) and \(\sigma\) and
calculated its correlation among all three weighting schemes. Although
visitation and efficiency (pollen deposition) produce moderately
different results, we found a very strong agreement between the order
produced by visitation and importance which is arguably a more accurate
metric of interspecific effects (Table
\ref{tab:tab-assumption-visitation}).

\begin{table}[!h]

\caption{\label{tab:tab-assumption-visitation}Spearman correlation coefficient matrix between the control capacity and the probability obtained by weighting links by the visitation, pollinator efficiency and pollinator importance.}
\centering
\fontsize{8}{10}\selectfont
\begin{tabular}[t]{lrrr}
\toprule
 & efficiency & importance & visitation\\
\midrule
\addlinespace[0.3em]
\multicolumn{4}{l}{\textbf{control capacity}}\\
\hspace{1em}efficiency & 1.00 & 0.37 & 0.38\\
\hspace{1em}importance & 0.37 & 1.00 & 0.86\\
\hspace{1em}visitation & 0.38 & 0.86 & 1.00\\
\addlinespace[0.3em]
\multicolumn{4}{l}{\textbf{superior probability}}\\
\hspace{1em}efficiency & 1.00 & 0.69 & 0.57\\
\hspace{1em}importance & 0.69 & 1.00 & 0.90\\
\hspace{1em}visitation & 0.57 & 0.90 & 1.00\\
\bottomrule
\end{tabular}
\end{table}

Altogether, the evidence supports the idea that visitation is a suitable
metric to estimate the mutual dependence of species pairs. First, it is
directly related to pollinator foraging. Second, it produces results
within our controllability framework that are consistent with plant
reproductive success (as estimated by the importance metric).

\clearpage

\section{Sensitivity to sampling}\label{undersampling}

Our approach is fundamentally based on the network structure. Often, the
majority of the interactions that make up this structure are weak; in
our networks, this means that most interactions are formed by a small
number of observed pollination visits, and therefore those weak
interactions are less conspicuous in the field than strong ones. To
strengthen the case of our approach, we, therefore, evaluated the
robustness of our results to simulated sampling limitations.

To do so, we removed a portion of the visits, for each network, and
calculated how three control metrics of the subsampled network compare
to those of the full network. Specifically, we calculated the difference
between the relative size of the minimum driver node set \(n_D\) of the
subsampled and the full network, the Spearman correlation between the
control capacities \(\phi\) and the probability of being a superior node
\(\sigma\) that were obtained for the empirical network and subsamples
of it. Finally, we also compared the control capacity of critical
species (\(\phi=1\)) across the different subsamples. We removed up to
20\% of the interactions in 1\% increments. We repeated the procedure
for each network to obtain a total sample size of \(n=400\).

Overall, the results of the sensitivity analysis indicate that our
approach is likely to still be useful in the absence of complete
sampling. Specifically, as the proportion of sampled interactions
decreased, the variability of \(n_D\) increased but was overall similar
to that obtained when using all available interactions (Figure
\ref{fig:fig-assumption-sampling-nd}). Similarly, the correlation
between \(\phi\) and \(\sigma\) of empirical and subsampled networks was
high even for extreme levels of undersampling (Figure
\ref{fig:fig-assumption-sampling-cor}). Importantly, critical species in
the empirical networks were also critical species in the subsampled
networks in a large majority of cases (Figure
\ref{fig:fig-assumption-sampling-crit}).

\begin{figure}[p]

{\centering \includegraphics{/Users/efc29/github/driver-species/paper/supp-info_files/figure-latex/fig-assumption-sampling-nd-1} 

}

\caption{Difference between the size of the minimum driver node set of the empirical network and the subsampled network. Lighter lines indicate the changes for each networks and the darker line indicates the overall trend.}\label{fig:fig-assumption-sampling-nd}
\end{figure}

\begin{figure}
\centering
\includegraphics{/Users/efc29/github/driver-species/paper/supp-info_files/figure-latex/fig-assumption-sampling-cor-1.pdf}
\caption{\label{fig:fig-assumption-sampling-cor}Spearman correlation between
the control capacity and the probability of being a superior nodes of
the empirical networks and subsampled networks. Lighter lines indicate
the changes for each networks and the darker line indicates the overall
trend.}
\end{figure}

\begin{figure}
\centering
\includegraphics{/Users/efc29/github/driver-species/paper/supp-info_files/figure-latex/fig-assumption-sampling-crit-1.pdf}
\caption{\label{fig:fig-assumption-sampling-crit}Control capacity of
critical species (those with \(\phi = 1\) in the empirical network) in
subsampled networks. Lighter lines indicate the changes for each
networks and the darker line indicates the overall trend.}
\end{figure}

\clearpage

\begin{table}

\caption{\label{tab:tab-controllability-model-results}Model selection table based on the Akaike Information Criterion for small samples for candidate models of size of the minimum driver node set ($n_D$) in the empirical networks. The evaluated variables are: the invasion status (inv), the network connectance ($C$), the network nestedness (NODF) the number of species ($n_S$) the network asymmetry ($AS_s$) or the interaction strength asymmetry ($AS_s$). Only candidate models with a weight $>0.01$ are shown.}
\centering
\fontsize{8}{10}\selectfont
\begin{tabular}[t]{rrrlrlrrrrr}
\toprule
(Intercept) & C & $AS_s$ & inv & $n_s$ & study & $AS_n$ & NODF & df & delta & weight\\
\midrule
0.89 & -0.40 & --- & --- & -0.20 & --- & -0.50 & -0.19 & 6 & 0.00 & 0.21\\
1.07 & --- & -0.11 & --- & --- & --- & -0.31 & --- & 4 & 1.33 & 0.11\\
0.95 & -0.38 & --- & --- & -0.20 & --- & -0.50 & --- & 5 & 1.46 & 0.10\\
1.03 & --- & -0.10 & --- & --- & --- & -0.31 & -0.15 & 5 & 2.23 & 0.07\\
0.84 & -0.42 & --- & --- & -0.22 & + & -0.46 & -0.20 & 7 & 2.91 & 0.05\\
1.02 & -0.11 & -0.13 & --- & --- & --- & -0.31 & --- & 5 & 2.91 & 0.05\\
0.98 & -0.11 & -0.12 & --- & --- & --- & -0.31 & -0.16 & 6 & 3.66 & 0.03\\
1.07 & --- & --- & --- & --- & --- & -0.38 & --- & 3 & 3.79 & 0.03\\
1.02 & --- & --- & --- & --- & --- & -0.38 & -0.18 & 4 & 3.90 & 0.03\\
0.91 & -0.34 & -0.04 & --- & -0.16 & --- & -0.45 & -0.18 & 7 & 4.09 & 0.03\\
0.97 & -0.30 & -0.06 & --- & -0.14 & --- & -0.43 & --- & 6 & 4.32 & 0.02\\
0.90 & -0.40 & --- & --- & -0.21 & + & -0.47 & --- & 6 & 4.43 & 0.02\\
0.88 & -0.41 & --- & + & -0.21 & --- & -0.51 & -0.20 & 7 & 4.65 & 0.02\\
1.06 & --- & -0.12 & --- & 0.02 & --- & -0.29 & --- & 5 & 4.75 & 0.02\\
1.07 & --- & -0.11 & + & --- & --- & -0.31 & --- & 5 & 4.93 & 0.02\\
1.07 & --- & -0.11 & --- & --- & + & -0.31 & --- & 5 & 4.95 & 0.02\\
0.95 & -0.38 & --- & + & -0.20 & --- & -0.51 & --- & 6 & 5.62 & 0.01\\
1.08 & --- & --- & --- & -0.04 & --- & -0.40 & --- & 4 & 5.97 & 0.01\\
\bottomrule
\end{tabular}
\end{table}

\begin{table}

\caption{\label{tab:tab-model-output-species-level}Summary table of the control capacity model with the smallest AICc.}
\centering
\fontsize{8}{10}\selectfont
\begin{tabular}[t]{lrrrr}
\toprule
term & estimate & std.error & statistic & p.value\\
\midrule
(Intercept) & -1.20 & 0.37 & -3.21 & 0.00\\
visitation strength & 0.29 & 0.16 & 1.86 & 0.06\\
contribution to nestedness & 0.15 & 0.14 & 1.05 & 0.29\\
asymmetry & 0.80 & 0.25 & 3.14 & 0.00\\
level (pol.) & 2.70 & 0.39 & 6.96 & 0.00\\
\addlinespace
contribution to nestedness:level (pol.) & 0.32 & 0.19 & 1.70 & 0.09\\
asymmetry:level (pol.) & -1.98 & 0.28 & -7.17 & 0.00\\
\bottomrule
\end{tabular}
\end{table}

\begin{table}

\caption{\label{tab:tab-var-importance}Variable importance for control capacity models. The importance is the sum of the weights of all models in which the variable was included.}
\centering
\fontsize{8}{10}\selectfont
\begin{tabular}[t]{lr}
\toprule
term & importance\\
\midrule
level (pol.) & 1.00\\
asymmetry & 1.00\\
asymmetry:level (pol.) & 1.00\\
visitation strength & 0.93\\
contribution to nestedness & 0.91\\
contribution to nestedness:level (pol.) & 0.79\\
visitation strength:level (pol.) & 0.39\\
degree & 0.28\\
degree:level (pol.) & 0.25\\
\bottomrule
\end{tabular}
\end{table}

\begin{figure}
\centering
\includegraphics{/Users/efc29/github/driver-species/paper/supp-info_files/figure-latex/fig-species-models-degree-1.pdf}
\caption{\label{fig:fig-species-models-degree}Relationship between degree
and (a) control capacity and (b) the probability of being a superior
node.}
\end{figure}

\begin{figure}
\centering
\includegraphics{/Users/efc29/github/driver-species/paper/supp-info_files/figure-latex/fig-correlogram-1.pdf}
\caption{\label{fig:fig-correlogram}Mean Spearman correlation coefficients
between control metrics (control capcity \(\phi\) and the likelihood of
being a superior node \(\sigma\)) and centrality metrics commonly used
to estimate a species' `keystoness' (closeness centrality \(l\) ,
betwenness \(w\), eigen centrality \(e\), page rank \(g\) and degre
\(k\)) in each of the networks. Purple and orange tiles indicate a
positive correlation or negative correlation respectively.}
\end{figure}

\clearpage 

\section*{References}\label{references}
\addcontentsline{toc}{section}{References}

\hypertarget{refs}{}
\hypertarget{ref-alarcon_congruence_2010}{}
Alarcón, R. (2010). Congruence between visitation and pollen-transport
networks in a California plantPollinator community. \emph{Oikos},
\emph{119}(1), 35--44.
doi:\href{https://doi.org/10.1111/j.1600-0706.2009.17694.x}{10.1111/j.1600-0706.2009.17694.x}

\hypertarget{ref-almeida-neto_straightforward_2011}{}
Almeida-Neto, M., \& Ulrich, W. (2011). A straightforward computational
approach for measuring nestedness using quantitative matrices.
\emph{Environmental Modelling \& Software}, \emph{26}(2), 173--178.
doi:\href{https://doi.org/10.1016/j.envsoft.2010.08.003}{10.1016/j.envsoft.2010.08.003}

\hypertarget{ref-ballantyne_constructing_2015}{}
Ballantyne, G., Baldock, K. C. R., \& Willmer, P. G. (2015).
Constructing more informative plant-pollinator networks: Visitation and
pollen deposition networks in a heathland plant community.
\emph{Proceedings of the Royal Society B: Biological Sciences},
\emph{282}(1814), 20151130.
doi:\href{https://doi.org/10.1098/rspb.2015.1130}{10.1098/rspb.2015.1130}

\hypertarget{ref-bascompte_asymmetric_2006}{}
Bascompte, J., Jordano, P., \& Olesen, J. M. (2006). Asymmetric
Coevolutionary Networks Facilitate Biodiversity Maintenance.
\emph{Science}, \emph{312}(5772), 431--433.
doi:\href{https://doi.org/10.1126/science.1123412}{10.1126/science.1123412}

\hypertarget{ref-bluthgen_specialization_2007}{}
Blüthgen, N., Menzel, F., Hovestadt, T., Fiala, B., \& Blüthgen, N.
(2007). Specialization, Constraints, and Conflicting Interests in
Mutualistic Networks. \emph{Current Biology}, \emph{17}(4), 341--346.
doi:\href{https://doi.org/10.1016/j.cub.2006.12.039}{10.1016/j.cub.2006.12.039}

\hypertarget{ref-Csardi2006a}{}
Csardi, G., \& Nepusz, T. (2006). The igraph software package for
complex network research. \emph{InterJournal, Complex Systems},
\emph{1695}(5), 1--9.

\hypertarget{ref-Gutin2013}{}
Gutin, G. (2013). Independence and Cliques. In P. Zhang (Ed.),
\emph{Handbook of Graph Theory} (Second Edi, pp. 475--489). Chapman and
Hall/CRC.
doi:\href{https://doi.org/10.1201/b16132-31}{10.1201/b16132-31}

\hypertarget{ref-king_why_2013}{}
King, C., Ballantyne, G., \& Willmer, P. G. (2013). Why flower
visitation is a poor proxy for pollination: Measuring single-visit
pollen deposition, with implications for pollination networks and
conservation. \emph{Methods in Ecology and Evolution}, \emph{4}(9),
811--818.
doi:\href{https://doi.org/10.1111/2041-210X.12074}{10.1111/2041-210X.12074}

\hypertarget{ref-liu_control_2016}{}
Liu, Y.-Y., \& Barabási, A.-L. (2016). Control principles of complex
systems. \emph{Reviews of Modern Physics}, \emph{88}(3).
doi:\href{https://doi.org/10.1103/RevModPhys.88.035006}{10.1103/RevModPhys.88.035006}

\hypertarget{ref-liu_controllability_2011}{}
Liu, Y.-Y., Slotine, J.-J., \& Barabási, A.-L. (2011). Controllability
of complex networks. \emph{Nature}, \emph{473}(7346), 167--173.
doi:\href{https://doi.org/10.1038/nature10011}{10.1038/nature10011}

\hypertarget{ref-neeman_framework_2010}{}
Ne'eman, G., Jürgens, A., Newstrom-Lloyd, L., Potts, S. G., \& Dafni, A.
(2010). A framework for comparing pollinator performance: Effectiveness
and efficiency. \emph{Biological Reviews}, no--no.
doi:\href{https://doi.org/10.1111/j.1469-185X.2009.00108.x}{10.1111/j.1469-185X.2009.00108.x}

\hypertarget{ref-rohr_structural_2014}{}
Rohr, R. P., Saavedra, S., \& Bascompte, J. (2014). On the structural
stability of mutualistic systems. \emph{Science}, \emph{345}(6195),
1253497--1253497.
doi:\href{https://doi.org/10.1126/science.1253497}{10.1126/science.1253497}

\hypertarget{ref-saavedra_nested_2016}{}
Saavedra, S., Rohr, R. P., Olesen, J. M., \& Bascompte, J. (2016).
Nested species interactions promote feasibility over stability during
the assembly of a pollinator community. \emph{Ecology and Evolution},
\emph{6}(4), 997--1007.
doi:\href{https://doi.org/10.1002/ece3.1930}{10.1002/ece3.1930}

\hypertarget{ref-vazquez_interaction_2005}{}
Vázquez, D. P., Morris, W. F., \& Jordano, P. (2005). Interaction
frequency as a surrogate for the total effect of animal mutualists on
plants: Total effect of animal mutualists on plants. \emph{Ecology
Letters}, \emph{8}(10), 1088--1094.
doi:\href{https://doi.org/10.1111/j.1461-0248.2005.00810.x}{10.1111/j.1461-0248.2005.00810.x}

\hypertarget{ref-zhang_input_2016}{}
Zhang, X., Lv, T., \& Pu, Y. (2016). Input graph: The hidden geometry in
controlling complex networks. \emph{Scientific Reports}, \emph{6}(1).
doi:\href{https://doi.org/10.1038/srep38209}{10.1038/srep38209}


\end{document}
