\documentclass[12pt]{letter}

\usepackage[britdate]{canterbury-letter}
\usepackage{times}
%\usepackage{letterbib}
\usepackage{geometry}
% \usepackage[round]{natbib}
\usepackage{graphicx}
%\graphicspath{ {../figures/ecology_figs/} }
\geometry{a4paper}
\usepackage[T1]{fontenc}
\usepackage[utf8]{inputenc}
\usepackage{authblk}
\usepackage[running]{lineno}
\usepackage{amsmath,amsfonts,amssymb}
\usepackage[margin=10pt,font=small,labelfont=bf]{caption}


\newenvironment{refquote}{\smallskip \begin{it}}{\end{it}\smallskip}

\name{Dr.\ Daniel B.\ Stouffer}
\position{Associate Professor}
\department{School of Biological Sciences}
\location{Private Bag 4800}
\telephone{+64 3 364 2729}
\email{daniel.stouffer@canterbury.ac.nz}
\url{http://stoufferlab.org}

% \date{14 junio 2017}

\begin{document}

\begin{letter}{
       %\vspace*{20pt}
       Prof. Spencer Barrett\\
       Editor in Chief\\
       Proceedings of the Royal Society B}

\opening{Dear Professor Barrett,}

We are submitting the manuscript entitled ``Quantifying the manageability of pollination networks in an invasion context'' to be considered for publication at \emph{Proceedings of the Royal Society B}. 

Central to this manuscript is the introduction, development, and application of the theoretical concept of \textbf{ecological manageability}.
Based on the patterns of interactions between species, this idea can provide two main pieces of information. 
First, it allows us to quantify the \textit{community manageability}---an indication of the inherent difficulty of controlling the abundances of all species in the community. 
Second it allows us to identify \textit{driver species}---those species whose manipulation exhibits a disproportionate ability to alter other species' abundances. 
Furthermore, our approach to identify driver species, and quantify their relative importance, is unique in that it is based directly on the aspects of network structure that determine the behavior of the community as a holistic dynamical system. 

Here, we use the concept of ecological manageability and the information it provides as a framework to study the changes induced by biotic invasions to pollination networks. 
Multiple empirical observations have suggested that the structural changes caused by invasions may lead to changes to ecosystem resilience. 
Our study is the first to use empirical data to provide direct theoretical support for this idea. 
Our results also highlight the extent to which some species---invaders in particular---can drive the dynamics of other species across their community, and also identify ``asymmetric dependence'' as the potential mechanism driving the interspecific differences.

Despite our focus, understanding how drivers of ecosystem change affect ecosystem dynamics is not exclusive to mutualistic communities. 
Indeed, we envisage the application of the concepts we introduce to answer a wide range of ecological questions in communities structured by different types of interactions.
As such, we have sought to highlight the methodology necessary for its implementation---an extension of recently developed tools at the interface of complex systems and control theory---throughout the manuscript. 

Our study fundamentally lies within ecological theory, yet it also has ramifications pertinent to ecological application. 
Indeed, we show in the paper that concepts like ``manageability'' have the potential to stimulate much needed research that can reduce the gap between theory and management.
Hopefully our study and more like it will set us on the track to informed, rather than hopeful, ecosystem interventions. 

Thank you in advance for your consideration of our manuscript at your journal.

\closing{Regards,}

\end{letter}

\end{document}
