\documentclass[10pt, a4paper]{letter}

\begin{document}

\begin{letter}{
       %\vspace*{20pt}
       Prof. Tim Coulson\\
       Editor in Chief\\
       Ecology Letters}

\opening{Dear Proffesor Coulson,}

We are submitting the manuscript entitled ``Biotic invasions reduce the manageability of pollination networks'' to be considered for publication at \emph{Ecology Letters}. 

Central to this manuscript is the introduction, development, and application of the theoretical concept of \textbf{ecological manageability}.
Based on the patterns of interactions between species, this idea can provide two main pieces of information. 
First, it allows to quantify the \textit{community manageability}---an indication of the difficulty of controlling the abundances of all species in the community. 
Second it allows to identify \textit{driver species}--- those with a disproportionate ability to alter other species' abundances.  
Here, we use this concept as a framework to study the changes caused by biotic invasions to pollination networks. 

Multiple empirical observations have suggested that the structural changes caused by invasions may lead to increased ecosystem resilience. 
Our study is the first to use empirical data to provide theoretical support for this idea. 
Furthermore, our approach to identify driver species and quantify their relative importance, is unique in that it is based on the aspects of network structure that determine the behavior of the community as a dynamic system. 
Our results highlight the extent to which some species---invaders in particular---can drive the dynamics of other species across their community, and also identify ``asymmetric dependence'' as the potential mechanism driving the interspecific differences.

Despite our focus, understanding how drivers of ecosystem change affect ecosystem dynamics is not exclusive to mutualistic communities. 
Indeed, we envisage the application of the concepts we introduce to answer a wide range of ecological questions in communities structured by different types of interactions.
As such, we have sought to highlight the methodology necessary for its implementation---an extension of recently developed tools at the interface of complex systems and control theory---thoroughout the manuscript. 

Our study fundamentally lies within ecological theory, yet it has ramifications pertinent to ecological application. 
We think that concepts like theoretical manageability have the potential to stimulate much needed research that shorten the gap between ecological theory and ecosystem management.
Hopefully setting us on the track to informed, rather than hopeful, ecosystem interventions. 

Lastly, please note that the enclosed work represents a novel contribution for all co-authors involved and no related work published, in press, or submitted during this or last year has been cited. 

\closing{Regards,}

Daniel B. Stouffer

\end{letter}

\end{document}