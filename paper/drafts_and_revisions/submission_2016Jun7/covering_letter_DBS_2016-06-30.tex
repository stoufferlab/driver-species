\documentclass[12pt, a4paper]{letter}

\usepackage[britdate]{canterbury-letter}
\usepackage{times}

\position{Associate Professor}
\department{School of Biological Sciences}
\location{Private Bag 4800}
\telephone{+64 3 364 2729}
\fax{+64 3 364 2590}
\email{daniel.stouffer@canterbury.ac.nz}
\url{http://stoufferlab.org}
\name{Daniel B. Stouffer}

\begin{document}

\begin{letter}{
       %\vspace*{20pt}
       Prof. Tim Coulson\\
       Editor in Chief\\
       Ecology Letters}

\opening{Dear Professor Coulson,}

We are submitting the manuscript entitled ``Biotic invasions reduce the
manageability of mutualistic networks'' {\bf update at end to match text} to
be considered for publication at \emph{Ecology Letters}.

Central to our manuscript is the introduction, development, and application of
the theoretical concept of \emph{ecological manageability}. Based on the
patterns of interactions between species, this idea can provide two main
pieces of information. First, it allows us to quantify \textit{community
manageability}---an indication of how difficult it is to control or manipulate
the abundances of all species in an ecological community. Second, it allows us
to identify the \textit{driver species}---those with a disproportionate
ability to alter other species' abundances.

Here, we use this concept as a framework to study the changes caused by biotic
invasions in pollination networks. Multiple empirical observations have
suggested that the structural changes caused by invasions may lead to
increased ecosystem resilience. However, our study is the first to use empirical data
to provide robust theoretical support for this idea. Furthermore, our
approach to quantify the relative importance of species is unique relative to the prevailing literature in that is
based on how structure underpins community dynamics while at the same
time having a direct link with management. Our results highlight the
extent to which some species---invaders in particular---can drive the
dynamics of other species across their community. It also identifies
``asymmetric dependence'' as the key factor driving these interspecific
differences.

Within the present manuscript, we develop the core ideas of community
manageability and driver species around the objective of using them to study
the consequences of biotic invasions in mutualistic systems like pollination
networks. Nevertheless, our framework and its potential application for
understanding how global-change drivers affect ecosystem dynamics are clearly
not exclusive to mutualistic communities. Indeed, we envisage the application
of the concepts we introduce to answer a wide range of ecological questions in
communities structured by different types of interactions. As such, we have
sought to highlight the methodology necessary for its implementation---an
extension of recently developed tools at the interface of complex systems and
control theory---throughout the manuscript.

% DBS: I'd cut the following paragraph entirely. Yes it's pertinent, but it's
% already in the abstract and conclusion and doesn't strike me as essential
% for the letter. Thoughts?

% In addition, although our study lies fundamentally within ecological theory,
% it has ramifications pertinent to ecological application. We think that
% concepts like theoretical manageability have the potential to stimulate much
% needed research that shorten the gap between ecological theory and
% management. Hopefully setting us on the track to informed, rather than
% hopeful, ecosystem interventions.

Lastly, please note that this manuscript represents a novel contribution for all co-authors involved and no related work published, in press, or submitted during this or last year has been cited. 

\closing{Regards,}

\end{letter}

\end{document}