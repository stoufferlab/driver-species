\documentclass[10pt, a4paper]{letter}

\begin{document}

\begin{letter}{
       %\vspace*{20pt}
       Prof. XX\\
       Chief Editor\\
       Ecology Letters}

\opening{Dear Proffesor}

We are submitting the manuscript entitled ``Biotic invasions reduce the manageability of mutualistic networks'' to be considered for publication at \emph{Ecology Letters}. 

Central to this manuscript is the introduction and development of the concept of \textbf{theoretical manageability}.
Based on the patterns of interactions between species, this idea can provide two main pieces of information. 
First, it allows to quantify the \textit{community manageability}---an indication of the difficulty to control the abbundances of all species in the community. 
Second it allows to identify the \textit{driver species}--- those with a disproportionate ability to alter other species' abundances.  
We use this concept as a framework to study the changes caused by biotic invasions to the community manageability and examine the role played by individual species. 

Multiple empirical observations have suggested that the structural changes caused by invasions may lead to increased ecosystem resilience. 
Our study is the first to use empirical data to provide explicit theoretical support to this idea. 
Furthermore, our approach to quantify the relative importance of species is unique at that is based on the structure that underpin the community dynamics while at the same time having a direct link with management. 
Not only our results highlight the ability that some species, and invaders in particular, have to drive the dynamics of other species in the community, but also identifies "asymmetric dependence" as the key factor driving these interspecific differences.

We developed the ideas of community manageability and driver species with the underlying motivation to study the consequences of biotic invasions in mutualistic systems. 
Nevertheless, understanding how drivers of ecosystem change affect ecosystem dynamics is not exclusive to mutualistic communities. 
Indeed, we envisage the application of the concepts we introduce to answer a wide range of ecological questions in communities structured by different types of interactions.
The methodology necessary for its implementation---an extension of recently developed tools at the interface of complex systems and control theory---is thoroughly detailed in the manuscript. 

In addition, although our study lies fundamentally within ecological theory, it has ramifications pertinent to ecological application. 
We think that concepts like theoretical manageability have the potential to stimulate much needed research that shorten the gap between ecological theory and management.
Hopefully setting us on the track to informed, rather than hopeful, ecosystem interventions. 

The enclosed work represents a novel work also for all co-authors involved and no work published, in press, or submitted during this or last year has been cited. 

\closing{Regards,}

Daniel B. Stouffer

\end{letter}

\end{document}